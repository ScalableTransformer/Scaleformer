\section{Background}

\subsection{The original multi-head attention mechanism}

The scaled dot-product attention proposed by
\citet{vaswani2017attention} transforms the vectorial embedding $\vec{Y}_i$ of a token, as a
 function of a sequence variable in size of other tokens $\vec{X}_j$. Where
 $\vec{Y}_i$ and $\vec{X}_j$ are all vectors of size $D$. A key
$\vec{K}_j$ and value $\vec{V}_j$ are attributed to each vector
$\vec{X}_j$, and query $\vec{Q}_i$ is attributed to $\vec{Y}_i$.
 The query keys and values are obtained by linear projection from dimension $D$ to $d$ using three matrices of learnable parameters. The transformed vector
 $\vv{y_i'}$ is a weighted sum of the $\vec{V}_j$. The weights are
 scores of matching between the query $\vec{Q}_i$ and the keys
$\vec{K}_j$, calculated as the dot product between the two vectors.
The weights are also \emph{softmaxed} to sum up to 1. 

The transformation of $L_Q$
vectors $\vec{Y}_i$ as a function of $L_K$ vectors $\vec{X}_j$ can be efficiently computed with matrix multiplications:

\begin{equation}
A = \mathrm{softmax}\left(\frac{Q \times K^T}{\sqrt{d}}\right) \times V
\end{equation}

With $Q$ a matrix of shape $(L_Q, d)$, $K$ a matrix of shape
 $(L_K, d)$ and $V$ a matrix of shape $(L_K, d)$. The $\sqrt{d}$
at the denominator is a scaling factor used to avoid saturation in the
exponential terms of the softmax function.

The multi-head attention performs $h$ different projections into spaces
 of dimension $d = D/h$. The resulting vector $\vec{Y}_i'$ is the
concatenation of the $h$ vectors $\vv{y_i'}$ obtained. Thus the
embedding dimension is preserved. Using multiple heads was found
beneficial by the authors over using a single head of dimension
$d = D$.

During training, the cross entropy of the $n^{th}$ predicted token is
 calculated assuming all previous tokens have been generated correctly.
 This enables to parallelize training completely as there is no
recurrence in the calculation process. However as the $n^{th}$ token should not depend of the
following tokens, the cells in the upper right corner of the score
matrix are set to $-\infty$ such that after the softmax they are equal
to 0, and the rows still sums up to 1.

\subsection{Improving scalability}

The original attention mechanism requires the computation of a score
matrix $Q \times K^T$ of shape $(L_Q, L_K)$, with complexity
$O(L_QdL_K)$. If the query and key sequence lengths are multiplied by
two, then the memory used and computation time are multiplied by 4. To
improve the scalability of the transformer with sequence length, several axis of research have been explored.

\citet{kitaev2020reformer} proposed
the Reformer's architecture, which uses an hash-bucketting algorithm to
reduce the complexity of the original multi head attention operation
from $O(L^2)$ to $O(L\log(L))$.

\citet{dai2019transformerxl} proposed the
Transformer-XL's architecture, which cuts the sequence in segments of
length L. The model predicts each stage of the current segment as a
function of the previous and current segment. All the segments are
computed sequentially with a recurrence mechanism. The complexity is linear
with sequence length, but the computation cannot be completely
 parallelized due to the recurrence mechanism.

Other publications explored using a sparse attention matrix, such as the
 Longformer by \citet{beltagy2020longformer} and the Big Bird model by
 \citet{zaheer2021big}. As each
token attends to a fixed number of all other tokens, the scalability is
improved. These sparse attention models however require custom
operations implemented in CUDA.

Some other works propose to modify the attention mechanism so that it's complexity scales linearly with sequence length. The Linformer by
\citet{wang2020linformer} projects the
 key and values onto a smaller sequence length dimension with matrix
 multiplication. It cannot however generalize to sequences longer than
 during training, as the weights of the projection for such tokens would
 be undefined.

\citet{shen2020efficient} proposed to
 replace the softmax attention score.
 $A = \mathrm{softmax}\left(\frac{Q \times K^T}{\sqrt{d}}\right) \times V$ is
changed into $A = \rho(Q) \times \rho(K)^T \times V$. With $\rho$
the softmax function along the embedding dimension. Thanks to matrix
multiplication commutativity, the order of the operations can be chosen.
If $Q$, $K$ and $V$ are of shape $(L_Q, d)$, $(L_K, d)$ and
$(L_K, d)$ respectively, the complexity of
$(\rho(Q) \times \rho(K)^T) \times V$ is $O(L_Q \times d \times L_K)$ whereas the complexity of
$\rho(Q) \times (\rho(K)^T \times V)$ is
$O\left(max(L_Q, L_k) \times d^2 \right)$. The right-side-first
operation is linear in complexity with sequence length. The shape of the
intermediate result matrix is also changed, allowing to scale better
in memory requirements as well. The original \emph{softmaxed} attention score
matrix was giving rows of positive scores that sum to 1. With this
change the elements of the score matrix remain positive as $\rho(Q)$
and $\rho(K)^T$ are matrices of positive values, but the rows of the score matrix does not sum up to 1. This work also does not give a linear complexity formulation for masked attention. If the right-side-first
scheme is adopted, the attention score matrix
$\rho(Q) \times \rho(K)^T$ is never explicitly computed, and can't be
masked.

Building on this idea of commutative attention function proposed by
\citep{shen2020efficient}, \citet{katharopoulos2020transformers} introduced their kernerlized attention function as:

\begin{equation}
A = \frac{\phi(Q) \times \phi(K)^T}{\sum_j \left( \phi(Q) \times \phi(K)^T \right)} \times V
\end{equation}

The function $\phi$ is applied element-wise and can be any positive
function, for example $\phi(x) = elu(x) + 1$. This attention is
row-wise normalized so that all rows of the score matrix are sets of
positive weights adding up to one. This preserves the objective of the
original softmaxed attention scores, while allowing to perform
operations in an optimal order.

The Performer by \citet{choromanski2021rethinking} exploits the same idea of a kernelized attention introduced
by \citet{katharopoulos2020transformers}, with an algorithm that better approximates softmaxed attention. Most
importantly they also give in annex a prefix sum algorithm to perform operations in the right-side-first order while giving the same result as masked left-side-first operation. Althought they give no insight in their paper as how the operation could be implemented without custom CUDA code.

In this work we will explicit an implementation of the right-side-first masked operation, with usual functions from neural network frameworks, that remains linear in complexity.

\subsection{Alternative positional encodings}

The original multi-head attention operation introduced by
\citet{vaswani2017attention} was
intrinsically invariant by token order permutation. As token position was
an important information for machine translation models, they encoded
the global position of each token in their embedding. Since then, some
modified attention mechanisms, that depend on relative tokens position,
have been proposed.

\citet{shaw2018selfattention} explored
modifying the attention mechanism so that it depends on the relative
distance between tokens. A second score matrix that is function of the
query and the query/key relative distance is added to the original score
matrix. $A = softmax\left(\frac{Q \times K^T}{\sqrt{d}}\right) \times V$ becomes
$A = \left(\frac{Q \times K^T + S_{rel}}{\sqrt{d}}\right) \times V$ with $S_{rel}$ of shape $(L_Q, L_K)$ defined as
${S_{rel}}_{ij} = \vec{Q_i} \cdotp \vec{RP}_{clip(i-j, -k, k)}$. Where $k$ is the attention horizon length and $\vec{RP}_n$ is one of
$2k+1$ relative positional embedding, vectors of size $d$.
\citet{shaw2018selfattention} and
\citet{huang2018music} observed that introducing this attention scheme improved performances.
The naive calculation of this term however has a complexity of
$O(L_QL_Kd)$. No algorithm was provided to linearize the complexity.

More recently \citet{liutkus2021relative} gives a stochastic positional encoding that is linear in
complexity with regards to sequence length. However the implementation
is complex and its stochastic nature requires that the operations be
repeated several times in parallel.

\citet{horn2021translational} noted that
the term $S^{rel} \times V$ can be computed with linear complexity for
the case where $RP_{-k} = RP_{k}$. However this is restraining as the
model can't make the difference between tokens before the attention
horizon or after.

In this work we will show that the computation of $S^{rel} \times V$
can also be done with linear complexity, without concession.

\endinput
