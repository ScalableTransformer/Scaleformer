\section{Linear Scalable Transformer
model}

As stated earlier, the proposed model proposes the replacement of the
scaled dot-product attention from original Transformer architecture by a
kernelized attention with relative positional encoding

The proposed model replaces the scaled-dot-product-attention by a
kernelized attention with RPE. Following the observations of
\citet{shaw2018selfattention} that
accumulating absolute positional encoding with relative positional
encoding yield no benefits, the positional encoding is also removed -
although for some specific applications it might be beneficial to
maintain it. The algorithm used to calculate each term is chosen
dynamically to occupy the least memory depending on the sequence lengths
and embedding dimensions - as memory usage is easier to evaluate
precisely than execution time.

In this work we have chosen the following formulation, with
$\phi(x) = elu(x) + 1$.

\begin{equation}
A = \frac{\left( \phi(Q) \times \phi(K^T) + S^{rel} \right)}{\sum_j \left( \phi(Q) \times \phi(K^T) + S^{rel} \right)} \times V
\end{equation}

This is essentially a combination of two terms: the kernelized attention
proposed by  \citet{katharopoulos2020transformers}, and the relative positional encoding proposed by
\citet{shaw2018selfattention}. The left
term is the score matrix of shape $(L_Q, L_K)$, with a denominator
which scales all rows so that they sum to 1. For the sake of the
implementation, the multiplication must be distributed as:

\begin{equation}
A = \frac{\left( \phi(Q) \times \phi(K^T) \times V \right) + \left( S^{rel} \times V\right)}{\sum_j \left( \phi(Q) \times \phi(K^T) \right) + \sum_j \left( S^{rel} \right)}
\end{equation}

The denominator can be easily calculated by applying the (naive or
linear complexity) algorithms with V replaced by a matrix of shape
$(L_K, 1)$ full of 1.

For each case (masked/bidirectional) the algorithm is chosen between
naive and linear complexity to occupy the least memory.

\begin{itemize}
\item
for the masked $Q \times K^T \times V$ term, the memory occupied by
the naive algorithm is $L_QL_K$ while the linear complexity
algorithm occupies $d^2 \times max(L_Q, L_K)$
\item
for the bidirectional $Q \times K^T \times V$ term, the memory
occupied by the naive algorithm is $L_QL_K$ while the linear
complexity algorithm occupies $d^2$
\item
for the $S_{rel} \times V$ term (masked and bidirectional), the
memory occupied by the naive algorithm is $L_QL_K$ while the linear
complexity algorithm occupies $L_Q \times (2k+1 + 4)$
\end{itemize}

\endinput
