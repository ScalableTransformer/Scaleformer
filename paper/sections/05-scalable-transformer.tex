\section{The Scaleformer}

The proposed model replaces the scaled-dot-product-attention by a
kernelized attention with RPE. Following the observations of
\citet{shaw2018selfattention} that
accumulating absolute positional encoding with relative positional
encoding yield no benefits, the positional encoding is also removed.

In this work we have chosen the following formulation, with $\phi(x) = exp(x)$.

\begin{equation}
A = \frac{\left( \phi(Q) \times \phi(K^T) + S^{rel} \right)}{\sum_j \left( \phi(Q) \times \phi(K^T) + S^{rel} \right)} \times V
\end{equation}

This is essentially a combination of two terms: the kernelized attention
proposed by  \citet{katharopoulos2020transformers}, and the relative positional encoding proposed by
\citet{shaw2018selfattention}. The left
term is the score matrix of shape $(L_Q, L_K)$, with a denominator
which scales all rows so that they sum to 1.

For the linear complexity implementation, the multiplication must be distributed as:

\begin{equation}
A = \frac{\left( \phi(Q) \times \phi(K^T) \times V \right) + \left( S^{rel} \times V\right)}{\sum_j \left( \phi(Q) \times \phi(K^T) \right) + \sum_j \left( S^{rel} \right)}
\end{equation}

The denominator can be easily calculated by applying the linear complexity algorithms with V replaced by a matrix of shape $(L_K, 1)$ full of 1, or by summing the rows of the score matrix for quadratic complexity algorithm.

\endinput
