\section{The Scaleformer}

The proposed model replaces the scaled-dot-product-attention  and absolute positional encoding by a kernelized attention with relative positional encoding.
In this work we have chosen the following formulation, with $\phi(x) = \mathrm{elu}(x) + 1$.

\begin{equation}
A = \frac{\left( \phi(Q) \times \phi(K^T) + S^{rel} \right)}{\sum_j \left( \phi(Q) \times \phi(K^T) + S^{rel} \right)} \times V
\end{equation}

This is essentially a combination of two terms, the kernelized attention
proposed by  \citet{katharopoulos2020transformers}, and the relative positional encoding proposed by
\citet{shaw2018selfattention}. The left
term is the score matrix of shape $(L_Q, L_K)$, with a denominator
which scales all rows so that they sum to 1.

For the linear complexity implementation, the multiplication must be distributed as:

\begin{equation}
A = \frac{\left( \phi(Q) \times \phi(K^T) \times V \right) + \left( S^{rel} \times V\right)}{\sum_j \left( \phi(Q) \times \phi(K^T) \right) + \sum_j \left( S^{rel} \right)}
\end{equation}

The denominator can be easily calculated by applying the linear complexity algorithms with V replaced by a matrix of shape $(L_K, 1)$ full of 1, or by summing the rows of the score matrix for quadratic complexity algorithm.

For compatibility with the kernelized attention, as the scores are not \emph{softmaxed} and must be positive, the function $\phi$ must be applied to $Q$ and $RP$ before calculating $S^{rel}$:

\begin{equation}
	S^{rel}_{ij} = \sum_l \phi(Q)_{il} \times \phi(RP)_{clip(i-j, -k, k), l}
\end{equation}

The linear complexity algorithms are a trade-off of quadratic complexity with sequence length for quadratic complexity with projection dimension $d$. Increasing expressiveness power of the model can however be done efficiently by increasing the number of heads. Both the naive and linear algorithm have linear complexity with the number of attention heads.

\endinput
