\section{Linear complexity RPE\protect\footnote{Relative Positional Encoding}}

In this section we will detail how the relative positional encoding
 proposed by \citet{shaw2018selfattention}
can in fact be computed with linear complexity. The $S^{rel}$ matrix of scores is computed as
$S^{rel}_{ij} = \vv{Q_i} \cdot \vv{RP}_{clip(i-j, -k, k)}$.

In figure \ref{fig:S_rel} the colors represent the index of the query, and the number the
index of the relative position.

The relative positional encoding term of the attention is calculated as $A^{rel} = S^{rel} \times V$. Each row of the $A^{rel}$ matrix is a weighted sum of the value vectors $V_i$. Each row of $S^{rel}$ is a set of weights.

One can observe in figure \ref{fig:A_rel_naive} that some weights are repeated several times in the
$S^{rel}$ matrix. Calculating the whole matrix can be avoided by
instead calculating all possible weights only once, with complexity
$O \left(L_Q\times d\times(2k+1)\right)$). As illustrated in figure \ref{fig:A_rel_linear}, the matrix multiplication can then be replaced by a sum of three terms. The grey squares represent some zero-padding. The first term is a cumulated sum of the value vectors that is weighted  by the before-horizon set of weights - complexity $O(\max(L_Q, L_K))$. The second term is a weighed sum of a moving window of the value vectors. Where the weights are the central diagonal of the $S^{rel}$ matrix - complexity $O(L_Q \times (2k-1) \times d)$. The last term is similar to the first term, but for the after-horizon set of weights  - complexity $O(\max(L_Q, L_K))$. The implementation is given in algorithm \ref{alg:A_rel}. The algorithm can be extended to masked attention by setting all vectors of the $RP$ matrix with strictly positive indexes to $\vec{0}$.

\endinput
